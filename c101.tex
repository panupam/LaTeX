\documentclass[10pt,a4paper]{article} 
\usepackage{amsmath}
\usepackage{amsfonts}
\usepackage{amssymb}
\title{Calculus}
\author{Pranav K Anupam \thanks{Thanks to many good book on Calculus from which this material is  inspired }}
\date{August 2019}

\begin{document}
\begin{titlepage}
\maketitle
\end{titlepage}
\tableofcontents
\clearpage
\begin{center}
\section{ Limit and Continuity}
%\hline
\end{center}
\subsection{Limit of a function}
\textbf{Definition } A number $A$ is called limit of a function $f$ at a point $a$ if for every $\varepsilon > 0$ , $\exists$ a $\delta>0$ such that $|f(x)-f(a)|<\varepsilon$ whenever $|x-a|<\delta$. If such number $A$ exists then it is unique.\par
\textbf{Right Hand Limit } A number $A$ is said to be the \emph{right hand limit} of $f(x)$ as $x \to a$ \emph{from the right} if ,for any arbitrary chosen $\varepsilon>0$ $\exists$ a corresponding $\delta>0$ such that $|f(x)-A|<\varepsilon$ $\forall x \in (a,a+\delta) $ \par 
We write it as $\displaystyle{\lim_{x \to a^+} f(x)}=A$ \par
\textbf{Left Hand Limit } A number $A$ is said to be the \emph{left hand limit} of $f(x)$ as $x \to a$ \emph{from the left} if ,for any arbitrary chosen $\varepsilon>0$ $\exists$ a corresponding $\delta>0$ such that $|f(x)-A|<\varepsilon$ $\forall x \in (a-\delta,a) $ \par 
We write it as $\displaystyle{\lim_{x \to a^-} f(x)}=A$ \par
\emph{Limit of $f(x)$ exists only if both the left hand and right hand limit exists and equal to each other otherwise we say that limit of function $f(x)$ at $a$ doesn't exists.}
\subsection{Some theorem about limits}
If $f$ and $g$ are two function such that $\displaystyle{\lim_{x \to a} f(x)}= A$ and $\displaystyle{\lim_{x \to a} g(x)}= B$ where $A$ and $B$ are finite, then, 
\begin{enumerate}
\item $\displaystyle{\lim_{x \to a}} (kf) x = kA$ where $k$ is a constant. 
\item $\displaystyle{\lim_{x \to a}} (f \pm g) x = A \pm B $
\item $\displaystyle{\lim_{x \to a}} (f g) x = A.B $
\item $\displaystyle{\lim_{x \to a}} (f / g) x = A/B $ provided $B \neq 0$

\end{enumerate}
\subsection{Infinite Limits}
A function $f$ is said to tends to infinity as $x \to a$ if for any positive number $N$, however large, $\exists$ a $\delta > 0 $ such that $f(x) > N$ $\forall x~\in~(a-\delta,a+\delta)$
\par \vspace{1em}
\textbf{Example }\#1 By using $\varepsilon-\delta$ method prove that 
\begin{enumerate}
\item $\displaystyle{\lim_{x \to 3}} (x^2 + 2x)= 15$
\item $\displaystyle{\lim_{x \to 0}} x^2\sin(1/x)=0$
\end{enumerate}
\textbf{Solution :} \par 1) According to definition of limit $|f(x)-A|$. i.e. \par 
\begin{equation}
\begin{split}
|x^2+2x-15| & =|x-3||x+5| \\ &\leq |x-3|(|x|+5) \\ & <|x-3|.2b \\ &\text{ where $b$ is fixed positive number greater than $|x|$ \& $5$}
\end{split} 
\end{equation}
Take arbitrary positive number $\varepsilon$ and chose $\delta=\varepsilon/2b$  then \\ by (1)
 $|x^2+2x-15|<\varepsilon$ whenever $0<|x-3|<\delta$.
Hence by definition of limit $\displaystyle{\lim_{x \to 3}} (x^2 + 2x)= 15$\\

2)  $|f(x)-A|$
\begin{equation}
|x^2\sin(1/x)-0|=|x^2\sin(1/x)|<|x^2|  
\end{equation}
Take arbitrary positive number $\varepsilon$ and choose $\delta=\sqrt{\varepsilon}$ then by (2)\\ $|x^2\sin(1/x)|<\varepsilon$ whenever $0<|x|<\delta$.\\ So by definition of limit  $\displaystyle{\lim_{x \to 0}} x^2\sin(1/x)=0$ 
 \hfill $\square$
 \\ 
 \textbf{Example} \#2 Evaluate the following limits 
 \begin{enumerate}
 \item $\displaystyle{\lim_{x \to 0} \frac{1-\cos x}{x}}$
 \item $\displaystyle{\lim_{x \to 0} \frac{\sin ax}{\sin bx}}$
 \item $\displaystyle{\lim_{x \to 0} \frac{\tan x -\sin x}{x^3}}$
 \item $\displaystyle{\lim_{x \to 0} \frac{3\sin^{-1} x}{4x}}$
 \item $\displaystyle{\lim_{x \to a} \frac{x^n-a^n}{x-a}}$
 \end{enumerate}
 \textbf{Solution} \\
 1) \begin{equation*}
 	\lim_{x \to 0} \frac{1-\cos x}{x}=\lim_{x \to 0} \frac{2\sin^2 x/2}{x} =\lim_{x \to 0} \frac{\sin x/2}{x/2}  \lim_{x \to 0} \sin x/2 =1\times0 =0
 \end{equation*}
 2) \begin{equation*}
 \lim_{x \to 0} \frac{\sin ax}{\sin bx} = \lim_{x \to 0} \frac{\frac{\sin ax}{x}}{\frac{\sin bx}{x}} = \frac{a}{b} \lim_{x \to 0} \frac{\frac{\sin ax}{ax}}{\frac{\sin bx}{bx}} = \frac{a}{b} \times \frac{1}{1}=\frac{a}{b}
 \end{equation*}
 Because  $\displaystyle{\lim_{x \to 0} \frac{\sin x}{x}=1}$  \par 
 3) 
 \begin{equation*}
 \lim_{x \to 0} \frac{\tan x -\sin x}{x^3}= \lim_{x \to 0} \frac{\sin x(1 -\cos x)}{x^3 \cos x}= \lim_{x \to 0} \frac{\sin x(2\sin^2x/2)}{x^3 \cos x}=\lim_{x \to 0} \frac{\sin x}{x}\lim_{x \to 0} \frac{2\sin^2x}{(x/2)^2\times 4}\lim_{x \to 0} \frac{1}{\cos x}  
 \end{equation*}
 $\displaystyle{=1\times\frac{1}{2}\times1= \frac{1}{2}}$
 
 4) Let $\sin^{-1}x=\theta$ as $x \to 0$ so does $\theta$ also $x=\sin \theta$
 \begin{equation*}
 \lim_{x \to 0}\frac{3\sin^{-1}x}{4x}= \lim_{\theta \to 0}\frac{3\theta}{4\sin\theta}=\frac{3}{4} $\\$($as $ \lim_{x \to 0} \frac{\sin x}{x}=1)
 \end{equation*} 
 \par 5) 
 \begin{equation*}
 \lim_{x \to a}\frac{x^n-a^n}{x-a}=\lim_{x \to a}\frac{(x-a)(x^{n-1}+x^{n-2}a+...+a^{n-1})}{x-a}=\lim_{x \to a}(x^{n-1}+x^{n-2}a+...+a^{n-1})=na^{n-1}
 \end{equation*}
 \textbf{Example }\#3  Prove or disprove $f(x)= x\sin (1/x) $; $x\neq0$ then $\lim_{x \to 0} f(x) =0$ \par \textbf{Solution } As $|sin 1/x|$  lies between 0 \& 1 \begin{equation*}
 \lim_{x \to 0^+} x\sin 1/x = 0\times[Finite Number] = 0
 \end{equation*} Similarly 
 \begin{equation*}
 \lim_{x \to 0^-} x \sin 1/x = 0 \times[Finite Number]=0
 \end{equation*} As \emph{Left Hand Limit} is equal to \emph{Right Hand Limit} $\lim_{x \to 0} f(x)$ exists and is equal to $0$ 
 \\ \textbf{Example }\#4 Discuss the existence of $\lim_{x \to 0}x\tan^{-1}1/x$. \par \textbf{Solution } Range of $\tan^{-1} x$ is $[-\pi/2,\pi/2]$ for $\forall x\in \mathbb{R}$ 
 \begin{equation*}
 \lim_{x \to 0^+} x \tan^{-1} 1/x = 0\times[Finite]=0
 \end{equation*}Similarly 
 \begin{equation*}
 \lim_{x \to 0^-} x \tan^{-1} 1/x = 0\times[Finite]=0
 \end{equation*} 
 Hence $\lim_{x \to 0} x\tan^{-1} 1/x$ exists and is equal to 0.
 \subsection{Continuity}
 A function $f$ of a real variable x is said to be continuous at $x=a$ if for any arbitrary chosen $\varepsilon>0$ , however small, we can find a corresponding $\delta > 0$ such that $|f(x)-f(a)|<\varepsilon$ for all value of $x$ for which $0<|x-a|< \delta$. \\ Comparing the definition with limit we can conclude that \emph{for a function to be continuous at $x=a$} 
 \begin{equation*}
 \lim_{x \to a} f(x) = f(a)
 \end{equation*} 
 When $\lim_{x \to a} f(x) \neq f(a)$ or when it doesn't exist or is infinite , the function is \emph{discontinuous} at $x=a$ 
\\ A function is said to be \emph{continuous from left} if $\lim_{x \to a^-} f(x)$ exists and is equal to $f(a)$.
\\ A function is said to be \emph{continuous from right} if $\lim_{x \to a^+} f(x)$ exists and is equal to $f(a)$.
\subsection{Some Theorem about continuity} If two function are continuous at a point, then there sum, difference, product and quotient will also be continuous at that point. \\ If $f$ and $g$ be two function which are continuous at $x=a$ then the following will also be continuous at $x=a$
\begin{itemize}
\item $f \pm g$ 
\item $ f g$
\item $k f$ (k = constant) 
\item $f/g$ where $\lim_{x \to a} g(x) = g(a) \neq 0$
\end{itemize} 
\subsection{Continuous Function} A function $f$ is said to be \emph{ continuous in the interval $(a,b)$} if it is continuous for every value of x in interval $(a,b)$. \\ A function $f$ is said to be \emph{bounded} in the interval $(a,b)$ if there exist number $A,B$ such that $$A\leq f(x) \leq B$$  for every value of $x$ in the interval $(a,b)$.\\ A function $f$ defined in finite interval is \emph{bounded} unless the limit of $f(x)$ at any point is infinite. Therefore a function continuous in $[a,b]$ will be bounded.
\\
\textbf{Ex }\#1 Show that $\sin x$ is continuous for every real value of x. \\
\textbf{Solution } Let $a$ be any value of $x$ then , 
\begin{equation*}
|\sin x -\sin a|=|2\sin\frac{x-a}{2}\cos\frac{x+a}{2}|\leq2|\sin\frac{x-a}{2}| 
\end{equation*}
Because $|\cos\frac{x+a}{2}|<1$ . \\ Now for any $\theta$ we have $|\sin\theta|\leq |\theta|$ therefore we can say that 
$$|\sin x- \sin a|\leq 2|\frac{x-a}{2}|=|x-a|$$
Therefore $|\sin x -\sin a| < \varepsilon$ if $|x-a|<\varepsilon$ This is true for any positive real number. Hence $\sin x$ continuous for every real value of $x$. \\
\textbf{Ex }\#2 Is  the following function continuous at $x=2$ ? 
\begin{equation*}
f(x)= \begin{cases}
1+x & \hspace{1cm}  x\leq 2\\
5-x& \hspace{1cm}  x>2
\end{cases}
\end{equation*}
\textbf{Solution } $$\lim_{x \to 2^+}f(x)= \lim_{x \to 2^+}5-x=3$$ 
$$\lim_{x \to 2^-}f(x)=\lim_{x \to 2^-}1+x=3$$
Since \emph{left hand limit} is equal to \emph{right hand limit} given function is continuous at $x=2$. \\
\textbf{Ex }\#3 Are following functions continuous at $x=0$? 
\begin{enumerate}
\item \begin{equation*}
f(x)= \begin{cases} 
\cos(1/x)& $when $ x\neq 0 \\
0 & $when $ x=0
\end{cases}
\end{equation*}
\item \begin{equation*}
f(x)=\begin{cases}
x\sin(1/x) & $when $ x\neq 0 \\
0 & $when $ x = 0
\end{cases}
\end{equation*}
\textbf{Solution } 1) As $x \to 0^+$ or $x \to 0^-$ the given $f(x)$ fluctuate between $[-1,1]$ as the function doesn't converge to a point as $x \to 0$ limit doesn't exists. \par 
2) $$ \lim_{x \to 0^+} f(x) = \lim_{x \to 0^+} x\sin(1/x) = 0\times[Finite] = 0$$
Similarly $$ \lim_{x \to 0^-} f(x) = \lim_{x \to 0^-} x\sin(1/x) = 0\times[Finite] = 0$$ As \emph{left hand limit \& right hand limit } are equal and are equal to $f(0)$ . Hence given function is continuous at $x=0$.
\end{enumerate} 
\textbf{Ex }\#4 Prove that the function $f(x)=\frac{1}{x}\sin^{-1}x$ for $x\neq 0$ , $f(0)=1$ is continuous at origin. \\
\textbf{Solution } $$ \lim_{x \to 0}\frac{1}{x} \sin^{-1} x=1$$ as $\lim_{x \to 0}f(x) = f(0)$ Hence given function is continuous at origin.
\\ \textbf{Ex }\#5 Examine the continuity of function $$f(x)=\frac{1}{1-e^x}$$ at $x=0$.\\
\textbf{Solution }
\end{document}