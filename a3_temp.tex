\documentclass[10pt]{letter}
\usepackage{amsmath}
\usepackage{amsthm}
\usepackage{amssymb}
\usepackage{fullpage}
%\usepackage{enumerate}
\usepackage{graphicx}
%\usepackage{hyperref}
%\usepackage[]{algorithm2e}

\newcommand{\abs}[1]{{\left\lvert #1 \right\rvert}}	
\newenvironment{solution}{\textbf{Solution:}}{\hfill$\square$}


\begin{document}

\begin{center}
Assignment 2: CS 201 (Fall 2018)
\end{center}

I pledge on my honor that I have not given or received any unauthorized assistance.

Name:        ~~~~~~~~~~~~~~~~~~~~~~~~~~~~~~~~~~~~~~   Roll No.:       \\

%If you discuss the question with someone, please mention their name on your assignment. Remember, you are allowed to discuss and not copy.
%You can use any theorem used in the class except if that theorem itself needs to be proven in the question.

%Please print on both sides while submitting the assignment.

\begin{enumerate}

\item (5) Let $G=(V,E)$ be an undirected graph with $m$ edges. We pick a subset $S$ of  
$V$ at random by choosing a vertex with probability $1/2$. What is the expected  
number of edges between $S$ and $V-S$?
Hint: Linearity of expectation.

\begin{solution}
%Write your solution here.
\end{solution}

\item (2+3) Prove the following.
\begin{itemize}
\item Let $P$ be an alternating path for a matching $M$. Consider $P$ as a set of edges.
Show that $M' = (M\setminus P) \cup (P \setminus M)$ is a matching such that $\abs{M'} \geq \abs{M}$.
\item Let $M$ be an $n\times n$ matrix with entries $0$ and $1$. You are given that every row and every column has exactly $k$ ones. 
Show that it is possible to find $n$ ones in the matrix such that no two of them are in the same row or same column.  
\end{itemize}  


\begin{solution}
%Write your solution here.
\end{solution}


\item (5) Give pseudocode for an efficient (polynomial in $\log{n}$) algorithm to calculate $a^b \mod n$ for some natural numbers $a,b,n$. 

\begin{solution}
%Write your solution here.
\end{solution}


\item (5) Define $\phi(m)$ to be the number of natural numbers less than $m$ and coprime to $m$ (their GCD with $m$ is $1$).
Let $n$ be an odd number, show that $\phi(n) = \phi(2n)$. 
 
\begin{solution}
%Write your solution here.
\end{solution}


\end{enumerate}

\end{document}


