\documentclass[10pt]{letter}
\usepackage{amsmath}
\usepackage{amsthm}
\usepackage{amssymb}
\usepackage{fullpage}
%\usepackage{enumerate}
\usepackage{graphicx}
%\usepackage{hyperref}
%\usepackage[]{algorithm2e}

\newenvironment{solution}{\textbf{Solution:}}{\hfill$\square$}


\begin{document}

\begin{center}
Assignment 2: CS 201 (Fall 2018)
\end{center}

I pledge on my honor that I have not given or received any unauthorized assistance.

Name:        ~~~~~~~~~~~~~~~~~~~~~~~~~~~~~~~~~~~~~~   Roll No.:       \\

If you discuss the question with someone, please mention their name on your assignment. Remember, you are allowed to discuss and not copy.
You can use any theorem used in the class except if that theorem itself needs to be proven in the question.

Please print on both sides while submitting the assignment.

\begin{enumerate}

\item (4) Write pseudocode to compute $\binom{n}{k}$. Show that its running time is polynomial.

\begin{solution}
%Write your solution here.
\end{solution}

\item (2+2+3) Remember that three numbers $0 < a\leq b\leq c$ can only form a triangle iff $a+b>c$. 
\begin{enumerate}
\item Show that there exist $6$ numbers in the interval $[1,8]$ such that no three of them can form a triangle.
\item Show that if three numbers exist in the range $[k,2k)$ then they form a triangle.
\item Show that there CAN'T exist $7$ numbers in the interval $[1,8]$ such that no three of them form a triangle.
\end{enumerate} 

\begin{solution}
%Write your solution here.
\end{solution}


\item (3+2) Define height of an element $x$ in a poset $P$ to be the highest number of elements in a chain whose last element is $x$ ($x$ is biggest in that chain).
\begin{enumerate}
\item Show that the set of all elements of height $h$ in a poset form an antichain.
\item If the largest chain in a poset has size $r$, show that $P$ can be partitioned into $r$ antichains.
\end{enumerate}  

\begin{solution}
%Write your solution here.
\end{solution}


\item (4) Find the number of primes less than $100$ using the principle of inclusion-exclusion.

\begin{solution}
%Write your solution here.
\end{solution}


\end{enumerate}

\end{document}


